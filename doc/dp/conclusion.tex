\chapter{Conclusion and Future Work}
\label{chap:conclusion}

We have developed an algorithm for reference-aided local genome assembly. We have compared it against well-known solutions, such as GATK, Fermikit or the traditional caller implemented in SAMTools. Although the comparison did not ended badly, there definitely is a space for improvements. Our implementation, stored on a DVD-ROM attached to this work, should be viewed as a prototype, not a solution ready for production. The implementation is not as fast as it should be, since no extra performance optimizations were made.

The algorithm should also be tested with more data sets, not just only at a 40 MB region of the first human chromosome. Additional testing may prevent over-tuning to certain test cases. One of the test cases should include variants with more phasing information. Our algorithm deduces the phasing information much more aggressively than others.

Probably the weakest part of the algorithm is the variant filtering step which should recognize and eliminate fake variants. More sophisticated methods than a binomial test and read coverage should be applied. For example, the algorithm does not utilize base qualities because our attempts to do so did not improve the results in any significant manner. Because of an absence of a good variant filtering mechanism, a constant quality value is assigned to all called variants since we are not able to reliably tell how much a certain variant may be fake. The binomial test seemed to be a reasonable solution, however, it succeeded only on variants with low read coverage, and even then its successes were limited.

The variant graph constructed during computation of genotype and phasing information, also may have a great potential in recognizing fake variants. Unfortunately, all attempts to do so failed so far.